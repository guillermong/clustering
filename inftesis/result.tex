\chapter[Resultados ]{Resultados }\label{ch:capitulo2}


En las pruebas se tom� una muestra no superior a 30 documentos ya que al aumentar la muestra el tiempo de ejecuci�n crece $\frac{n^3}{2}$, donde $n$ es la cantidad de documentos. La muestra se obtuvo aleatoriamente de la colecci�n de documentos, como la muestra es insignificante en comparaci�n al tama�o de la muestra es muy probable que ning�n documentos perteneciera a un documento de la misma versi�n. Este problema origina que la mayor�a de los documentos pertenezca a un solo grupo y el resto solamente es representado por 1 documento. Tambi�n cabe mencionar que la elecci�n de la cantidad de agrupaciones es arbitraria, pero la cantidad de agrupaciones es una variable importante al momento de obtener buenos resultados en las agrupaciones, en este caso las pruebas se realizaron con un n�mero fijo de agrupaciones para observar el comportamientos de otras variables que afectan a las agrupaciones.

%--------------------------------%

En la tabla \ref{Resultado algoritmo de agrupamiento aleatorio con 10 cluster.}  muestra los resultados  de cada m�todo con una cantidad de 10 agrupaciones, en cada agrupaci�n se muestra el resultado de la compresi�n. El \textit{M�todo 1}  utiliza el algoritmo de agrupaci�n aleatoria, se tiene que en cada grupo se mantiene una carga en memoria balanceada que es uno de los objetivos deseados en la memoria. En los m�todos siguientes se utiliza el algoritmo de agrupaci�n propuesto pero modificando algunas variables para observar su comportamiento.


%--------------------------------%

El \textit{M�todo 2}  se puede observar una mejora de la compresi�n equivalente al  45\% de memoria del resultado en el algoritmo de agrupamiento aleatorio, aqu� la muestra es de 30 documentos. En t�rminos de balance en la carga de memoria, este m�todo es ineficiente ya que la mayor parte de la carga se concentra solamente en un grupo. Esto se debe a que en el momento de crear los grupos con las muestras, la mayor parte de las muestra quedan solamente en un grupo dejando a las dem�s con pocas muestras de representaci�n.

%--------------------------------%

En el \textit{M�todo 3} se balancea la cantidad de muestras en cada agrupaci�n. Para esto cada grupo no tendr� una muestra de documentos superior a 3 en un universo de 30 documentos. Con esto se busca balancear la cantidad de documentos en cada agrupaci�n.
El resultado de la compresi�n utilizando el \textit{M�todo 3} es equivalente al 55\% de memoria del resultado en el compresi�n en el algoritmo de agrupamiento aleatorio, que sigue siendo una mejor alternativa, pero comparando con los resultado del \textit{M�todo 2} se paga un costo al balancear las muestras en los grupos, de un 24\% m�s de memoria del resultado en el \textit{M�todo 2}.

%--------------------------------%

En el \textit{M�todo 4}, se hace la misma prueba que en el m�todo anterior  pero se agrega una condici�n de que el tama�o en espacio de memoria no supere un l�mite, el limite en esta caso es el tama�o en memoria de la colecci�n de datos divido por la cantidad de agrupaciones, que se especifica en la formula \ref{eq3}, con esta medida se asegura que en todos los grupos tienen aproximadamente la misma cantidad de cadenas de texto.  El resultado del \textit{M�todo 4} es el equivalente al 70\% de memoria del resultado en el algoritmo de agrupamiento aleatorio que sigue siendo una mejora, pero nuevamente pagando un costo, con respecto al \textit{M�todo 2} aumenta 55\% m�s de memoria, incluso mayor que en el \textit{M�todo 3}, pero con mejores resultados en el balance de la carga en memoria. 


%------------------------------------------------------%


\begin{table}[H]
\begin{center}
\resizebox{15cm}{!} {

\begin{tabular}{|p{3cm}|p{3cm}||p{3cm}||p{3cm}||p{3cm}|}

\hline
Clusters & Metodo 1(KiB)  & Metodo 2(KiB)  & Metodo 3(KiB)  & Metodo 4(KiB) \\
\hline
Cluster 1 & 6.530 & 1.089 & 5.595 & 5.168 \\
\hline
Cluster 2 & 6.648 & 1.442 & 7.102 & 8.271 \\
\hline
Cluster 3 & 6.491 & 1.020 & 3.406 & 3.831 \\
\hline
Cluster 4 & 6.756 & 2.851 & 3.465 & 3.353 \\
\hline
Cluster 5 & 6.540 & 857 & 4.523 & 3.570 \\
\hline
Cluster 6 & 6.456 & 1.455 & 2.729 & 4.343 \\
\hline
Cluster 7 & 6.562 & 756  & 2.678 & 4.349 \\
\hline
Cluster 8 & 6.486 & 1.106 & 2.748 & 3.239 \\
\hline
Cluster 9 & 6.471 & 16.551 & 2.473 & 4.306 \\
\hline
Cluster 10 & 6.596  & 2.245 & 1.318 & 5.209 \\
\hline
Total  & 65.510 & 29.371 & 36.036 & 45.639 \\
\hline
\end{tabular}
}
\end{center}
\caption{Resultado algoritmo de agrupamiento aleatorio con 10 cluster.}

\label{Resultado algoritmo de agrupamiento aleatorio con 10 cluster.}

\end{table}	
