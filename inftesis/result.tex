\chapter[Resultados ]{Resultados }\label{ch:capitulo2}
\fpar

\parindent=0ptEn las pruebas se tomo una muesta no superior a 30 documentos ya que al aumetnar la muestra el tiempo de ejecucion crece exponencialmente*. La muestra se obtuvo aleatoriamente de la coleccion de documentos, como la muestra es insignificante en comparacion al tama�o de la muestra es muy probable que ningun documentos perteneciera a un documento de la misma version. Este problema orgina que la mayoria de los documentos pertecescan a un solo grupo y el resto solamente es representado por 1 documento.
\vspace{0.5cm}


%--------------------------------%

En la tabla \ref{Resultado algoritmo de agrupamiento aleatorio con 10 cluster.}  muestra el resultado del algoritmo de agrupamiento aleatorio formando 10 clusters.



\begin{table}[H]
\begin{center}
\resizebox{8cm}{!} {

\begin{tabular}{|p{3cm}|p{3cm}|}

\hline
Clusters & Tama�o(KiB) \\
\hline
Cluster 1 & 19.445 \\
\hline
Cluster 2 & 19.445 \\
\hline
Cluster 3 & 19.445 \\
\hline
Cluster 4 & 19.445 \\
\hline
Cluster 5 & 19.445 \\
\hline
Cluster 6 & 19.445 \\
\hline
Cluster 7 & 19.445 \\
\hline
Cluster 8 & 19.445 \\
\hline
Cluster 9 & 19.445 \\
\hline
Cluster 10 & 19.445 \\
\hline
Total  & 194.450 \\
\hline
\end{tabular}
}
\end{center}
\caption{Resultado algoritmo de agrupamiento aleatorio con 10 cluster.}

\label{Resultado algoritmo de agrupamiento aleatorio con 10 cluster.}

\end{table}	



%--------------------------------%



La tabla \ref{Resultado algoritmo de agrupamiento aleatorio con 10 cluster.} con 10 clusters y 30 documetnos de muestra, se puede observar una gran mejora con respecto al algoritmo de agrupamiento aleatorio.

\begin{table}[H]
\begin{center}
\resizebox{8cm}{!} {

\begin{tabular}{|p{3cm}|p{3cm}|}

\hline
Clusters & Tama�o(KiB) \\
\hline
Cluster 1 & 1.089 \\
\hline
Cluster 2 & 1.442 \\
\hline
Cluster 3 & 1.020 \\
\hline
Cluster 4 & 2.851 \\
\hline
Cluster 5 & 857 \\
\hline
Cluster 6 & 1.455 \\
\hline
Cluster 7 & 756 \\
\hline
Cluster 8 & 1.106 \\
\hline
Cluster 9 & 16.551 \\
\hline
Cluster 10 & 2.245 \\
\hline
Total  & 29.371 \\
\hline
\end{tabular}
}
\end{center}
\caption{Resultado algoritmo de agrupamiento aleatorio con 10 cluster.}

\label{Resultado algoritmo de agrupamiento aleatorio con 10 cluster.}
\end{table}	



%--------------------------------%


La tabla \ref{Resultado algoritmo de agrupamiento aleatorio con 10 cluster balanceando las muestras.} con 10 clusters y 30 documentos de muestra pero cada grupo no tendra una muestra de documentos superior a 3. Con esto se busca balancear los grupos.
En comparacion con el algoritmo de agrupamiento aleatorio sigue siendo una mejor alternativa pero comparando con los resulado de la tabla \ref{Resultado algoritmo de agrupamiento aleatorio con 10 cluster.} aumenta.

\begin{table}[H]
\begin{center}
\resizebox{8cm}{!} {

\begin{tabular}{|p{3cm}|p{3cm}|}

\hline
Clusters & Tama�o(KiB) \\
\hline
Cluster 1 & 5.595 \\
\hline
Cluster 2 & 7.102 \\
\hline
Cluster 3 & 3.406 \\
\hline
Cluster 4 & 3.465 \\
\hline
Cluster 5 & 4.523 \\
\hline
Cluster 6 & 2.729 \\
\hline
Cluster 7 & 2.678 \\
\hline
Cluster 8 & 2.748 \\
\hline
Cluster 9 & 2.473 \\
\hline
Cluster 10 & 1.318 \\
\hline
Total  & 36.036 \\
\hline
\end{tabular}
}
\end{center}
\caption{Resultado algoritmo de agrupamiento aleatorio con 10 cluster balanceando las muestras.}

\label{Resultado algoritmo de agrupamiento aleatorio con 10 cluster balanceando las muestras.}
\end{table}	


%--------------------------------%


La tabla \ref{Resultado algoritmo de agrupamiento aleatorio con 10 cluster balanceando las muestra y clusters.} , se hace la misma prueba pero se agrega una condicion de que el tama�o en espacio de memoria no supere un limite 

\begin{table}[H]
\begin{center}
\resizebox{8cm}{!} {

\begin{tabular}{|p{3cm}|p{3cm}|}

\hline
Clusters & Tama�o(KiB) \\
\hline
Cluster 1 & 5.168 \\
\hline
Cluster 2 & 8.271 \\
\hline
Cluster 3 & 3.831 \\
\hline
Cluster 4 & 3.353 \\
\hline
Cluster 5 & 3.570 \\
\hline
Cluster 6 & 4.343 \\
\hline
Cluster 7 & 4.349 \\
\hline
Cluster 8 & 3.239 \\
\hline
Cluster 9 & 4.306 \\
\hline
Cluster 10 & 5.209 \\
\hline
Total  & 45.639 \\
\hline
\end{tabular}
}
\end{center}
\caption{Resultado algoritmo de agrupamiento aleatorio con 10 cluster balanceando las muestra y clusters.}

\label{Resultado algoritmo de agrupamiento aleatorio con 10 cluster balanceando las muestra y clusters.}
\end{table}	



