\chapter[Introducci�n]{Introducci�n}\label{ch:capitulo1}

\section{Antecedentes generales }\label{chsub:Antecedentes}

 �La informaci�n tiene l�mites? �Somos capaces de guardar toda esta informaci�n? Hoy en d�a la informaci�n crece a pasos agigantados; cada d�a aparecen nuevos contenidos provenientes de p�ginas web, redes sociales, aplicaciones m�viles y de nuevas tecnolog�as como internet de las cosas, que son capaces de generar una gran cantidad de informaci�n. 

\vspace{0.5cm} 
\parindent=30pt Adem�s, esta informaci�n puede ir cambiando con el pasar del tiempo y en algunos casos, es necesario ser capaz de guardar el historial de cambios de esta. Ejemplos de aplicaciones que tienen estos requerimientos son por ejemplo: Wikipedia, una enciclopedia online colaborativa[5], y git, un manejador de versiones utilizado principalmente para almacenar c�digo de fuente[6]. Herramientas tradicionales de almacenamiento y versionamiento no son capaces de manejar una gran cantidad de datos. 

\vspace{0.5cm} 
\parindent=30pt Frente a este escenario, almacenar los datos de una aplicacion masiva es cada vez menos viable usando un solo computador. Por ejemplo, la empresa Backblaze, desarrolla una aplicaci�n que genera una copia de seguridad en la nube a muy bajo costo, usa una granja de servidores para almacenar los datos, repartiendo la informaci�n entre un conjunto de computadores que forman un sistema distribuido[2].

\vspace{0.5cm} 
Esta tesis consiste en estudiar una alternativa basada en clustering para repartir la informaci�n de manera inteligente entre varios computadores, haciendo uso de una m�trica de similitud basada en el contenido de la informaci�n para luego comprimirla, y as� de esta forma, se espera mejorar la compresi�n, idealmente manteniendo un balance en la carga de almacenamiento [3]. Clustering es un algoritmo que genera agrupaciones de objetos seg�n un criterio, se pretende con esto separar los datos de tal manera que en cada agrupaci�n se tenga un espacio parecido con respecto a los dem�s y que al utilizar cierto compresor sea mucho m�s eficiente en t�rmino de espacio que solamente separar los datos de manera aleatoria.



%=================================================ANTECEDENTES Y MOTIVACION========================================%

