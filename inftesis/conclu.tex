\chapter[Trabajo futuro ]{Trabajo futuro }\label{ch:capitulo4}
\fpar


En esta primera parte de la memoria se tuvo un proceso de estudio de los distintos algoritmos de agrupamiento m�s utilizados, mecanismos que permiten medir la similitud entre cadenas de caracteres y mecanismos para comprimir datos. Luego se implemento un algoritmo de agrupamiento aplicando lo estudiado y obteniendo los primeros resultados.

Evaluando los resultados se puede concluir en esta primera etapa de la memoria, que agrupando las cadenas de caracteres de manera inteligente, se obtiene mejores resultados que agrup�ndolos aleatoriamente, pero al intentar balancear la carga de espacio en cada agrupaci�n se paga un costo al comprimir. Parte importante para obtener una buena agrupaci�n es la medida de similitud, para agrupar grandes cantidades de cadenas de caracteres es necesario que la medida de similitud entre dos cadenas de caracteres sea r�pida, ya que cada  cadena de texto debe compararse con todos los del sampling. La ventaja de este algoritmo son que los resultados son determinista, es decir, cuantas veces se ejecuta el algoritmo para una misma colecci�n siempre entrega el mismo resultado, entonces al momento de obtener las agrupaciones con el sampling es posible asignar cadenas de caracteres en varios procesos, rebajando el tiempo de ejecuci�n.

El plan de trabajo para la segunda etapa de la memoria consistir� en intentar mejorar los resultados de la compresi�n de la colecci�n de datos obtenidos en esta primera parte, como por ejemplo, cambiando las distintas variables que influyen en el agrupamiento de las cadenas de caracteres, como el numero de agrupaciones, o tambi�n mejorando la implementaci�n del algoritmo de agrupamiento.