\chapter[Conclusiones]{Conclusiones }\label{ch:capitulo4}
\fpar


Se tuvo un proceso de estudio de los distintos algoritmos de agrupamiento m�s utilizados, mecanismos que permiten medir la similitud entre cadenas de caracteres y mecanismos para comprimir datos. Luego se implemento un algoritmo de agrupamiento aplicando lo estudiado y obteniendo los primeros resultados.

Evaluando los resultados se puede concluir en esta primera etapa de la memoria, que agrupando las cadenas de caracteres de manera inteligente, se obtiene mejores resultados que agrup�ndolos aleatoriamente, pero al intentar balancear la carga de espacio en cada agrupaci�n se paga un costo al comprimir. Parte importante para obtener una buena agrupaci�n es la medida de distancia, para agrupar grandes cantidades de cadenas de caracteres es necesario que la medida de distancia entre dos cadenas de caracteres sea r�pida, ya que cada  cadena de texto debe compararse con todos los del sampling. La ventaja de este algoritmo son que los resultados son determinista siempre que se mantenga el mismas muestras, es decir, cuantas veces se ejecuta el algoritmo para una misma colecci�n siempre entrega el mismo resultado, entonces al momento de obtener las agrupaciones con las muestras es posible asignar cadenas de caracteres en varios procesos, rebajando el tiempo de ejecuci�n.

Es importante aclarar que el algoritmo todavia puede seguir mejorando, en el analisis de los resultados se observo que la compres�on cambia dependiendo de las muestras iniciales y entre m�s muestras se tomen mejoran los resultados hasta llegar a un limite, pero con un mayor costo en el tiempo de ejecuci�n.

