\chapter[Conclusi�n ]{Conclusi�n}\label{ch:capitulo4}
\fpar

Se puede concluir que agrupando las secuencias de strings de manera inteligente se obtiene mejores resultados que agrup�ndolos aleatoriamente, pero intentar balancear la carga de espacio en cada agrupaci�n se paga un costo al comprimir. Parte importante para obtener una buena agrupaci�n es la medida de similitud, para agrupar grandes cantidades de string es necesario que la medida de similitud entre dos string sea r�pido ya que cada secuencia de string debe compararse con cada secuencia del sampling. La ventaja de este algoritmo son que los resultados son determinista, es decir, cuantas veces se ejecuta el algoritmo para una misma colecci�n siempre entrega el mismo resultado, entonces al momento de obtener las agrupaciones con el sampling es posible asignar strings en varios procesos rebajando el tiempo de ejecuci�n.

El plan de trabajo para la segunda etapa de la tesis consistir� en intentar mejorar los resultados de la compresi�n de la colecci�n de datos obtenidos en esta primera parte, por ejemplo, cambiando las distintas variables que influyen en agrupaci�n de strings como el numero de agrupaciones, para obtener mejores resultados en la compresi�n de las agrupaciones.
