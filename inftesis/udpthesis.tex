%
% ---------------------------------------------------------------
% template para escribir tesis/memorias en
% la Universidad Diego Portales
% ---------------------------------------------------------------
%
\documentclass[thesis,final]{udpbook}
%
\usepackage[latin1]{inputenc}    % esto NO es portable
\usepackage[spanish]{babel}
\selectlanguage{spanish}


\usepackage[T1]{fontenc}
\usepackage{amsmath}

\usepackage{amsmath} %%% para split
\usepackage{makeidx,color}  % allows for indexgeneration
\usepackage{graphicx}
\usepackage{subfigure} % subfiguras


\usepackage{tabulary}

\usepackage{fixltx2e}

\usepackage{mdframed}

\usepackage{dsfont}

\newcommand{\algorithmicbreak}{\textbf{break}}
\newcommand{\BREAK}{\STATE \algorithmicbreak}
% Default fixed font does not support bold face
\DeclareFixedFont{\ttb}{T1}{txtt}{bx}{n}{12} % for bold
\DeclareFixedFont{\ttm}{T1}{txtt}{m}{n}{12}  % for normal

% Custom colors
\usepackage{color}
\definecolor{deepblue}{rgb}{0,0,0.5}
\definecolor{deepred}{rgb}{0.6,0,0}
\definecolor{deepgreen}{rgb}{0,0.5,0}

\usepackage{listings}

% Python style for highlighting
\newcommand\pythonstyle{\lstset{
language=Python,
basicstyle = \ttm\tiny,
otherkeywords={self},             % Add keywords here
keywordstyle=\ttb\tiny\color{deepblue},
emph={MyClass,__init__},          % Custom highlighting
emphstyle=\ttb\color{deepred},    % Custom highlighting style
stringstyle=\color{deepgreen},
frame=tb,                         % Any extra options here
breaklines=true,
showstringspaces=false            % 
}}


% Python environment
\lstnewenvironment{python}[1][]
{
\pythonstyle
\lstset{#1}
}
{}

% Python for external files
\newcommand\pythonexternal[2][]{{
\pythonstyle
\lstinputlisting[#1]{#2}}}

% Python for inline
\newcommand\pythoninline[1]{{\pythonstyle\lstinline!#1!}}

\usepackage{color}
\usepackage{rotating}
\usepackage{longtable}
\usepackage{float}

\usepackage{algorithm}
\usepackage{algorithmic}
\usepackage{array}
\usepackage{eqparbox}

\renewcommand\algorithmiccomment[1]{%
  \hfill\#\ \eqparbox{COMMENT}{#1}%
}
\usepackage[nottoc]{tocbibind}
\makeatletter
\renewcommand{\ALG@name}{Algoritmo}
\renewcommand{\listalgorithmname}{List of \ALG@name s}
\makeatother
%\usepackage{fancyvrb}
%\DefineVerbatimEnvironment{code}{Verbatim}{fontsize=\small}
%\DefineVerbatimEnvironment{example}{Verbatim}{fontsize=\small}





%\providecommand{\abs}[1]{\lvert#1\rvert} 
%\providecommand{\norm}[1]{\lVert#1\rVert} 


%
% ----------------------------------------------------------------
% defina su escuela
%
\udpEscuela{Escuela Ingenier�a Inform�tica}{Facultad de Ingenier�a}
%
% ---------------------------------------------------------------
% Comienzo del documento
% ---------------------------------------------------------------
\usepackage{hyperref}
\hypersetup{%
    pdfborder = {0 0 0}
}
\usepackage[numbered]{bookmark}
\begin{document}
\frontmatter
%
% ----------------------------------------------------------------
% pagina de titulo
% ----------------------------------------------------------------
%
\begin{titlepage}
\title{Compresi�n de datos distribuida basada en clustering}
\author{Guillermo Andr�s Navarro Giglio}
\degreedoc{Memoria}{t�tulo}{Ingeniero Civil en Inform�tica y Telecomunicaciones}
\chairperson{Francisco Claude-Faust}      %
\committee{Roberto Konow}     % separados por \\
\date{Julio, 2015}	% (Ej: Enero, 2006)
\end{titlepage}                      %
%
% ----------------------------------------------------------------
% pagina de firmas
% ----------------------------------------------------------------
% (maximo seis firmas)
%
\begin{signaturepage}                   %
\approval{Francisco Claude-Faust\\Profesor gu�a}        %
\approval{Roberto Konow\\Comit�}%
\end{signaturepage}                     %
%
% ----------------------------------------------------------------
% pagina dedicatoria
% ----------------------------------------------------------------
% separadas por \\ si es mas de una linea
%
\begin{dedicatory}                      %
Dedico esta memoria a todas las personas.                %
\end{dedicatory}                        %
%
% ----------------------------------------------------------------
% Indices de materia, figuras y tablas
% ----------------------------------------------------------------
% paginas de contenido (indice de materias), lista de figuras
% y lista de tablas.
%
\tableofcontents                        % tabla de contenido
\listoffigures                          % �ndice de figuras
\listoftables                           % �ndice de tablas
%
% ----------------------------------------------------------------
% pagina de agradecimientos
% ----------------------------------------------------------------
%
%\begin{acknowledgment}                  %


%\end{acknowledgment}                    %
%
% ----------------------------------------------------------------
% pagina de abstract
% ----------------------------------------------------------------
%
%\begin{abstract}                        %
%This report tries to find an efficient mechanism to distribute a collection of data and then get the maximum compressibility maintaining a balance with regard to the use of resources on multiple servers. First the importance of the documents currently version explained and it is necessary to distribute the data, then a mechanism for distribution of data collection is implemented and finally a conclusion of the results is formed. The latter allows us to bring our program forward for the second stage of this report.
%\end{abstract}                          %
%
% ----------------------------------------------------------------
% pagina de resumen
% ----------------------------------------------------------------
%
\begin{resumen}                         %


Hoy en d�a la informaci�n crece r�pidamente con nuevos contenidos provenientes de p�ginas Web, redes sociales, aplicaciones m�viles entre otros. Cada vez es m�s dif�cil manejar esta gran cantidad de datos, demandando mayores recursos para las empresas y transform�ndose en un importante desaf�o en el futuro. Para esto, se debe buscar nuevos mecanismos que permitan de manera eficiente almacenar esta gran cantidad de datos.

Est� memoria intenta buscar un mecanismo eficiente de distribuir una colecci�n de datos para maximizar la compresibilidad y mantener un balance en lo que respecta al uso de recursos en m�ltiples servidores. Primero, se explica la importancia de los documentos versionados en la actualidad y por que se hace necesario distribuir los datos, luego se implementa un mecanismo de distribuci�n de la colecci�n de datos y finalmente se forma una conclusi�n de los resultados obtenidos. Esto �ltimo nos permite plantear nuestro programa de avance para la segunda etapa de esta memoria.



\end{resumen}                           %
%
% ----------------------------------------------------------------
% Fin de las paginas iniciales
% ----------------------------------------------------------------
%
\cleardoublepage                        %
\mainmatter                             %
                                        %
% ----------------------------------------------------------------
% Cap�tulos y secciones del documento
% ----------------------------------------------------------------
% aca se incluyen los archivos con el texto de los capitulos
% (Ej.: cha-intro.tex es el archivo con un capitulo)
%
\chapter[Introducci�n]{Introducci�n}\label{ch:capitulo1}

\section{Antecedentes generales }\label{chsub:Antecedentes}

 �La informaci�n tiene l�mites? �Somos capaces de guardar toda esta informaci�n? Hoy en d�a la informaci�n crece a pasos agigantados; cada d�a aparecen nuevos contenidos provenientes de p�ginas Web, redes sociales, aplicaciones m�viles y de nuevas tecnolog�as como Internet de las cosas, que son capaces de generar una gran cantidad de informaci�n. 

\vspace{0.5cm} 
\parindent=30pt Adem�s, esta informaci�n puede ir cambiando con el pasar del tiempo y en algunos casos, es necesario ser capaz de guardar el historial de cambios de esta. Ejemplos de aplicaciones que tienen estos requerimientos son por ejemplo: Wikipedia, una enciclopedia online colaborativa~\cite{wiki}, y Git, un manejador de versiones utilizado principalmente para almacenar c�digo de fuente~\cite{git}. Herramientas tradicionales de almacenamiento y versionamiento no son capaces de manejar una gran cantidad de datos de manera eficiente o incluso pr�ctica  que generan mayores costos, transform�ndose en un verdadero desaf�o para las empresas. 

\vspace{0.5cm} 
\parindent=30pt Frente a este escenario, almacenar los datos de una aplicaci�n masiva es cada vez menos viable usando un solo computador. Empiezan a aparecer nuevas soluciones a este problema, por ejemplo, la empresa Backblaze, desarrolla una aplicaci�n que genera una copia de seguridad en la nube a muy bajo costo, usa una granja de servidores para almacenar los datos, repartiendo la informaci�n entre un conjunto de computadores que forman un sistema distribuido~\cite{blackblaze}. Se deben buscar nuevos mecanismos que logren de manera eficiente almacenar los datos aprovechando los recursos, que se traduce en una disminuci�n en los costos de las empresas.

\vspace{0.5cm} 
Esta memoria consiste en estudiar una alternativa basada en clustering para repartir la informaci�n de manera inteligente entre varios computadores, haciendo uso de una m�trica de similitud basada en el contenido de la informaci�n para luego comprimirla, y as� de esta forma, se espera mejorar la compresi�n, idealmente manteniendo un balance en la carga de almacenamiento~\cite{handhookcompresion}. Clustering es una t�cnica que genera agrupaciones de objetos seg�n un criterio, se pretende con esto separar los datos de tal manera que en cada agrupaci�n se tenga un espacio parecido con respecto a los dem�s y que al utilizar cierto compresor sea mucho m�s eficiente en t�rmino de espacio que solamente separar los datos de manera aleatoria. 


\section{Conceptos B�sicos }\label{chsub:Objetivos}

En esta secci�n se explican conceptos b�sicos que se utilizaran m�s adelante y que ayudara a la comprensi�n del trabajo realizado.

\subsection{String o cadena de caracteres }\label{String o cadena de texto}

Una cadena de caracteres $S = \{s_{1},s_{2} \ldots, s_{l}\}$ es una secuencia de s�mbolos de alg�n alfabeto $\sum$ en particular de un tama�o $l$. 

\subsection{Coleccion de documentos}\label{Coleccion de documentos}

Una colecci�n de documentos es un conjunto de $\varrho$ documentos, representado como $C= \{d_{1},\ldots,d_{\varrho} \}$, donde los documentos son cadenas de caracteres $S$.

\subsection{Entrop�a de un texto}\label{Entrop�a de un texto}

La compresi�n de datos es la reducci�n del volumen de datos, sin embargo comprimir datos tiene un l�mite definido por la entrop�a, la entrop�a es la informaci�n nueva que se define como la cantidad total de datos menos su redundancia.

La entrop�a emp�rica de orden cero $H_{0}$ es el n�mero promedio de bits necesarios para representar cada s�mbolo del texto $T$ de tama�o $n$, est� definida como\cite{profe}:

  \begin{equation}
  \begin{aligned}
		H_{0}(S) &=\sum_{i=0}^{\sigma-1} \frac{n_{i}}{n}\log \frac{n}{n_{i}} 
  \end{aligned}\label{entropia0}
\end{equation}

donde el alfabeto $\sum =\{c_{0}, \ldots, c_{\sigma}\}$ de tama�o $\sigma$ y $n_{i}$ es el n�mero de ocurrencias de caracteres $c_{i}$ en $T$. 

\subsection{Punto}\label{Punto}
\sloppy{Punto: se define como un vector de $n$ dimensiones, $P= \{ v_{1}, v_{2},...., v_{n} \}$. Un punto $P \in R^{n}$ se dice que es un punto de cluster para un subconjunto $A$ si para cada $\delta > 0 $ tenemos $B(P;\delta)\cap A \not= \emptyset $, 
donde ${B(P;\delta ) = \{x \in R^{n} | \parallel x-a\parallel < \delta \} }$.}

\subsection{Medida de similitud}\label{Medida de similitud}

La medida de similitud es una funci�n real que cuantifica la similitud entre dos puntos, es lo opuesto a la medida de distancia. Se define la medida de similitud para un punto $x_{i}$ e $y_{j}$ como $s(i, j) = -||i - j||^2$ donde $||x - y||^2$ es la distancia euclidiana al cuadrado~\cite{similitud}.Ver cap�tulo \ref{ch:capitulo3} para una explicaci�n m�s detallada.\\

\subsection{Agrupamiento}\label{Agrupamiento}

Agrupamiento es el acto de formar grupos de puntos $M=\{ P_{1},\ldots, P_{\vartheta}\}$ de $\vartheta$ puntos, siguiendo alguna m�trica como puede ser la medida de similitud o de distancia, ver cap�tulo~\ref{ch:capitulo3} para una explicaci�n m�s detallada.



\section{Objetivos }\label{chsub:Objetivos}

El objetivo general de esta memoria es maximizar la compresibilidad de la colecci�n  agrup�ndolos en funci�n de su similitud de contenido, para esto se quiere  implementar y evaluar un mecanismo que represente colecciones versionadas de datos de forma distribuida.


\subsection{Objetivo espec�ficos}\label{chsub:Objetivos espec�ficos}


\begin{itemize}
  \item Realizar un estudio de mecanismos de clustering  para agrupar conjuntos de cadenas de caracteres.
  \item Generar un repositorio de datos distribuido basado en clustering.
  \item Medici�n de la efectividad de usar clustering previo a la compresi�n para el almacenamiento distribuido de datos, en t�rminos de la compresibilidad de la colecci�n.
\end{itemize}

\section{Plan de trabajo }\label{chsub:Plan de trabajo}



En la tabla ~\ref{Plan de trabajo} se presenta el plan de trabajo. Este plan tiene algunas alteraciones m�nimas en las fechas y tareas con respecto al plan original.

\begin{table}[H]
\begin{center}
\resizebox{15cm}{!} {

\begin{tabular}{|p{5cm}|p{3cm}||p{3cm}||p{4cm}||}

\hline
Tarea & Fecha inicio & Fecha final & Entregables \\
\hline
Anteproyecto & 20-03-2015   & 08-04-2014  & Anteproyecto \\
\hline
Estudiar y probar agrupamientos & 08-04-2015   & 30-04-2015  & Resultados experimentales en un set de datos preliminar \\
\hline
Implementar un separador de datos basado en agrupamientos & 30-04-2015   & 10-05-2015  & Sistema de agrupamiento funcionando \\
\hline
Documentaci�n & 10-05-2015 & 25-07-2015 & Tesis I \\
\hline
Mejorar algoritmo de agrupamiento & 25-07-2015 & 17-09-2015  & Algoritmo de agrupamiento \\
\hline
Realizar pruebas y comparaciones & 17-09-2015   & 15-10-2015  & An�lisis de Resultados \\
\hline
Documentaci�n final & 15-10-2015   & 01-12-2015  & Tesis II \\
\hline
\end{tabular}
}
\end{center}
\caption{Plan de trabajo.}

\label{Plan de trabajo}

\end{table}	


\section{Metodolog�a }\label{chsub:Metodologia}

La metodolog�a de desarrollo de la memoria se describe en los siguientes pasos:

\begin{enumerate}
  \item Se estudiar� el estado del arte de los distintos mecanismos de agrupamiento.
  \item Se analizar� y dise�ar� un algoritmo de agrupamiento para cadenas de caracteres.
  \item Se implementar� un algoritmo de agrupamiento. 
  \item Se realizar�n pruebas peque�as para comprobar su correcto funcionamiento. Si las pruebas iniciales son satisfactorias se realizan las pruebas con una colecci�n de datos mayor en el servidor.
  \item Se eval�an los resultados obtenidos, en base a los resultados se eval�a como mejorar el algoritmo implementado y se vuelve a repetir el proceso desde el segundo paso, para ir mejorando los resultados.
  \item
\end{enumerate}

Se contar� con un repositorio Git que contendr� el c�digo fuente desarrollada y las fuentes de la memoria misma.





%\chapter[Objetivos]{Objetivos}\label{ch:capitulo2}

\section{Objetivos }\label{chsub:Objetivos}

El objetivo general de esta memoria es implementar y evaluar un mecanismo que represente colecciones versionadas de datos de forma distribuida y que estos se encuentren agrupados en funci�n de su similitud de contenido de forma de maximizar la compresibilidad de la colecci�n.

Las colecciones que se utilizar�n en la memoria son Wiki-ES, Wiki-EN y el Kernel de Linux.

\section{Objetivo espec�ficos}\label{chsub:Objetivos espec�ficos}


\begin{itemize}
  \item Realizar un estudio de mecanismos de clustering  para agrupar conjuntos de Strings.
  \item Generar la implementaci�n de un repositorio de datos distribuido basado en clustering.
  \item Medici�n de la efectividad de usar clustering previo a la compresi�n para el almacenamiento distribuido de datos, en t�rminos de la compresibilidad de la colecci�n.
\end{itemize}

Donde String es cualquier cadena de s�mbolos ya sea texto o binario.





\chapter[Estado del arte]{Estado del arte}\label{ch:capitulo3}
\fpar


\section{Algoritmo de agrupamiento }\label{Algoritmo de agrupamiento1}

El algoritmo de agrupamiento es un proceso en la cual se tiene un conjunto de puntos y se crean grupos de puntos a partir de una medida de similitud, en la mayor�a de los algoritmos de agrupamiento se asume un espacio euclidiano. 

El espacio euclidiano es un espacio geom�trico que se cumplen los axiomas de Euclides, se deben cumplir tres reglas entre las distancias de dos puntos en el espacio euclidiano:

\begin{itemize}
  \item La distancia entre los puntos nunca es negativo y solamente es 0 consigo mismo.
  \item La distancia es simetrica, es decir, no importa si se calcula la distancia del punto $x$ a $y$ o de $y$ a $x$.
  \item La distancia obedece a la desigualdad del triangulos; la distancia  de x a y a z  no puede menor a la distancia de z a x. 
\end{itemize}

Los algoritmos de agrupamiento se pueden dividir en dos grupos, jer�rquicos  y no jer�rquicos. En los algoritmos de agrupamiento jer�rquicos pueden ser aglomerativos o divisivos.

Cuando es aglomerativo, cada punto en el espacio euclidiano representa un grupo y en cada iteraci�n se unen los grupos hasta llegar a los n�meros de grupos deseados, en cambio, los divisivos todos los puntos se encuentran en un s�lo grupo y en cada iteraci�n se dividen los grupos. El siguiente pseudocodigo~\ref{jerarquico} muestra el algoritmo de agrupamiento aglomerativo. Primero se debe determinar cu�ndo detener el algoritmo, una opci�n puede ser hasta tener el numero deseados de grupos.

\begin{algorithm}
\begin{algorithmic}
\WHILE{No es tiempo para detenerse}
\STATE Elegir los dos grupos m�s cercanos;
\STATE Unir ambos grupos en uno s�lo;
\ENDWHILE
\end{algorithmic}
\caption{Algoritmo Jerarquico aglomerativo}\label{jerarquico}
\end{algorithm}

Este algoritmo de agrupamiento es muy lento en caso de que la colecci�n de datos sea muy grande, existen algoritmo de agrupamiento para colecciones grandes, como CURE~\cite{superlibro}. 

CURE asume un espacio euclidiano, el algoritmo de CURE empieza tomando una peque�a muestra de la colecci�n principal y usa cualquier algoritmo de agrupamiento que asuma el espacio euclidiano sobre la muestra, por ejemplo, los algoritmo jer�rquicos serian una buena opci�n. Luego selecciona en cada grupo formado en la muestra, una peque�a cantidad de puntos que se llamaran ``puntos representativos'', estos puntos ser�n los que est�n m�s alejados del grupo y entre ellos mismos, la cantidad de puntos representativos es libre de elegirse. Despu�s los puntos representativos se mueven al centro del grupo un porcentaje, por ejemplo, 10\%.

La siguiente etapa consiste en unir los grupos que tengan un punto representativo cerca de un punto representativo de otro grupo, la distancia entre ambos puntos representativos para unir ambos grupos es libre de elegirse. Por �ltimo los puntos de la colecci�n se comparan con los puntos representativos y se asigna  al grupo que tenga la menor distancia.

En los algoritmo de agrupamiento no jerarquicos, \textit{k-means}~\cite{superlibro},  es unos de los algoritmo m�s utilizado y simple de entender.

K-means como la mayor�a de los algoritmos de agrupamiento asume un espacio euclidiano  y tambi�n asume el n�mero de agrupos conocidos. El n�mero de grupos se puede determinar de distintas maneras, como por ejemplo, por prueba y error o con conocimiento previo de las caracter�sticas de las observaciones.

Dado $k$ grupos se genera $k$ centros de grupos o centroides iniciales,los centroides son vectores de las medias de las caracter�sticas de todas las observaciones dentro de cada grupo. Los centroides se pueden asignar de distintas maneras, una opci�n es asignar observaciones aleatoriamente de un conjunto de observaciones. Luego se itera los siguientes pasos:


\begin{algorithm}
\begin{algorithmic}
\STATE C conjunto de $k$ centroides; 
\WHILE{true}
\STATE Para cada observacion calcular la distancia a todos los $C_{i}$;\COMMENT{$0<i<k$}
\STATE Las observaciones se asignan al $C_{i}$ con la media m�s cercana;
\STATE $C_{i}$ se recalculan las medias con las nuevas observaciones agregadas;
\IF{Todos los $C_{i}$ no cambian}
\STATE	Termina 
\ENDIF
\ENDWHILE
\end{algorithmic}
\caption{Algoritmo K-means}\label{kmeans2}
\end{algorithm}

Se itera hasta que converga, es decir, ya no se asignan nuevas observaciones a los grupos o los centroides ya no se mueven.

La mayor�a de los algoritmos  de agrupamiento asume un espacio euclidiano para medir la similitud entre los vectores, pero en el caso de las cadenas de caracteres no son puntos que se puedan representar en el espacio euclidiano. 

Para poder solucionar este problema se debe buscar un mecanismo para medir la similitud entre las cadenas de caracteres. Existen varias medidas de similitud para determinar la similitud entre cadenas de caracteres, en la siguiente secci�n se nombran algunas de las m�s utilizadas junto con la medida de similitud que se utiliz� en la implementaci�n del algoritmo de agrupamiento.


\section{Medida de similiud para Cadenas de caracteres }\label{Medida de similitud}

Unos de los puntos m�s importante al momento de implementar un algoritmo de agrupamiento es la medida de similitud entre los puntos, en este caso los puntos representan las cadenas de caracteres y se debe buscar una medida de similitud que logre una buena calidad. Una buena calidad en la medida de similitud  es lograr representar las cadenas de caracteres similares con una distancia peque�a y las cadenas de caracteres que no son similares en distancias mayores.


Existen varios m�todos para calcular la similitud entre una cadena de caracteres y otra como por ejemplo: 


\begin{itemize}
  \item Distancia de Edici�n: Se tiene dos cadenas de caracteres \(A\) y \(B\).La distancia de edici�n es la cantidad m�nima de insertar, sustituir o eliminar necesarios para transformar \(A\)  en \(B\). Con distancia de edici�n se logra una buena calidad en la similitud, pero presenta una desventaja: el algoritmo de edici�n de distancia requiere de tiempo  de $O(n \times m)$, donde $n$ y $m$  son el largo de ambas secuencias de strings. Por ejemplo para las cadenas ``abracadabra'' y ``alabaralabarda'', se necesitan 7 operaciones. La formula~\ref{eqeditdist} calcula la distancia de edici�n $d_{mn}$~\cite{distedicion}:
  
   \begin{equation}
  \begin{aligned}d_{i0} &= \sum_{k=1}^{i} w_\mathrm{del}(b_{k}), & & \quad  \text{for}\; 1 \leq i \leq m \\
d_{0j} &= \sum_{k=1}^{j} w_\mathrm{ins}(a_{k}), & & \quad \text{for}\; 1 \leq j \leq n \\
d_{ij} &= \begin{cases} d_{i-1, j-1} & \text{for}\; a_{j} = b_{i}\\ \min \begin{cases} d_{i-1, j} + w_\mathrm{del}(b_{i})\\ d_{i,j-1} + w_\mathrm{ins}(a_{j}) \\ d_{i-1,j-1} + w_\mathrm{sub}(a_{j}, b_{i}) \end{cases} & \text{for}\; a_{j} \neq b_{i}\end{cases} & & \quad  \text{for}\; 1 \leq i \leq m, 1 \leq j \leq n.  \end{aligned}\label{eqeditdist}
\end{equation}

donde $a = a_1\ldots a_n$ y $b = b_1\ldots b_m$ .
  
  
  
  
  \item Jaccard: Es una medida de similitud que est� definido por el tama�o de la intersecci�n de dos secuencias divido por el tama�o de la uni�n de ambas secuencias, un ejemplo en la medida de similitud entre las secuencias  ``night'' y ``nacht'' es de 0.3. Se tiene el conjunto S y T la formula para determinar el similitud jaccard es ~\ref{eq01}
  
  \begin{equation}\label{eq01}
	Jaccard = \frac{(S\cap T)}{(S\cap T)},
\end{equation}
  
  \item Distancia Hamming: Se tiene dos cadenas de caracteres de igual tama�o y se calcula cantidad de sustituciones necesarios para transformar una cadena de caracteres en otra.
  Por ejemplo ``night'' y ``nacht'' se necesita 2 sustituciones. 
  La formula de Distancia Hamming es \ref{eq234}
  \begin{equation}
  \begin{aligned}
		D_{h}(s_{i},s_{j}) &=\sum_{k=1}^{m} \delta(s_{ik},s_{jk}) \\
		\text{donde }\; \delta(x,y) &= \begin{cases} 0 \text{ si x = y }\; \\ 1 \text{ si x}\; \not = y \end{cases}
  \end{aligned}\label{eq234}
\end{equation}

  $s_{i}$ y $s_{j}$ son cadenas de caracteres y $m$ es el largo de las cadenas de caracteres.

\end{itemize}



En esta memoria se quiere una medida de distancia basada en compresi�n, la medida de similitud implementada para el algoritmo de agrupaci�n es una distancia de compresi�n utilizando alg�n m�todo de compresi�n como lzma, gzip o bzip, ver \ref{Metodos de compresi�n de datos}. Se aplica la siguiente f�rmula para obtener la $Similitud$ \ref{eq1}

\begin{equation}\label{eq1}
Similitud = \frac{(d_{1+2} - d_{2})}{d_{1}},
\end{equation}

donde $d_{1}$ es el tama�o comprimido del documento m�s grande,$d_{2}$ es el tama�o comprimido del documento m�s peque�o, y $d_{1+2}$ es el tama�o comprimido de la uni�n de $d_{1}$ y $d_{2}$ sin comprimir.

El valor de la variable $d_{1+2}$ va a depende de la similitud de las cadenas de caracteres comprimidas, si las cadens de caracteres tienen un grado de similitud la compresi�n ser� mucho m�s efectiva ya que se necesita un diccionario mucho menor para comprimir, al contrario ocurre cuando las cadenas de caracteres son muy distintas entre s�.

Entre m�s peque�o el valor de la variable $Similitud$, significa que ambas cadenas de caracteres son muy similares, y entre m�s grande los valores significa que las cadenas de caracteres son diferentes. En comparaci�n con la distancia edici�n est� obtiene una buena calidad de similitud, pero el tiempo de ejecuci�n es lineal. Mejorando el tiempo $O(n\times m)$ de la distancia de edici�n. Esto hace una buena opci�n al momento de seleccionar una medida de similitud para grandes colecciones de datos. Esta medida de similitud como entrega solamente un valor $R^{n}$ es de una sola dimensi�n.

Por ejemplo, si se utiliza el compresor LZ78~\cite{profe} en la secuencia $S_{1}$ = \textit{`abracadabra'}, $S_{2}$ = \textit{`abracadadah'} y la suma $S_{1+2}$=\textit{`abracadabraabracadadah'}, al aplicar la formula \ref{eq1} se obtiene el valor 0,71.

Ahora si cambiamos la segunda secuencia a $S_{3}$= 'casasyperro', la cual no tiene similitud con la primera secuencia, se obtiene el valor 0.77. De esta forma se observa que con menor similitud entre los strings, mayor es el valor, entonces al tener strings que son similares el valor se acerca al 0 .


\section{Metodos de compresi�n de datos }\label{Metodos de compresi�n de datos}

La compresi�n de datos es la reducci�n del volumen de datos. Existe dos tipos de compresores: la compresi�n sin p�rdida y la compresi�n con p�rdida.

En nuestro caso estudiaremos el caso particular LZ78~\cite{profe2}, LZ78 es un compresor sin p�rdida basado en diccionario, es un algoritmo greedy adaptativo. En el diccionario guarda un �ndice y un car�cter, el �ndice entrega la posici�n de su  prefijo de una secuencia y el car�cter es el �ltimo car�cter de la subcadena. El algoritmo empieza recorriendo la cadena de texto desde el principio, car�cter por car�cter, revisa si el car�cter nuevo ya se encuentra en el diccionario o si pertenece a una subcadena del diccionario, si ya se encuentra sigue con el siguiente car�cter, en caso de que no se encuentra, ingresa al diccionario ese nuevo car�cter seguido con el �ndice de su prefijo.

Por ejemplo, se tiene la cadena de texto S= ``abracadabra'', el resultado de la compresi�n con LZ78 se muestra en la tabla\ref{clz78}.


\begin{table}[H]
\begin{center}
\resizebox{2cm}{!} {

\begin{tabular}{|p{1cm}|p{1cm}|}

\hline
N�& S1 \\
\hline
1 & <0,a>  \\
\hline
2 & <0,b> \\
\hline
3 & <0,r> \\
\hline
4 & <1,c>\\
\hline
5 & <1,d> \\
\hline
6 & <1,b> \\
\hline
7 & <3,a> \\
\hline
\end{tabular}
}
\end{center}
\caption{Ejemplo LZ78.}

\label{clz78}

\end{table}	





\chapter{Algoritmo de agrupamiento propuesto}\label{Algoritmo de agrupamiento propuesto}

\section{Proceso de agrupamiento}\label{Proceso de agrupamiento}

El Proceso que se utiliz� para agrupar una colecci�n de datos se describe en los siguientes pasos:

\begin{enumerate}
  \item Primero se obtiene la colecci�n de datos que puede ser cualquier cadena de caracteres, por ejemplo, secuencias de ADN o informaci�n de Wikipedia.
  
  \item Se elige un algoritmo de agrupamiento y se ejecuta sobre la colecci�n. Cuando termina de ejecutar, la colecci�n se encontrar� repartida en diferentes directorios que representan los grupos formados por el algoritmo de agrupamiento.

  \item Por �ltimo, se comprime cada directorio utilizando alg�n m�todo de compresi�n.
  
\end{enumerate}


\section{Algoritmo de agrupamiento implementado}\label{Algoritmo de agrupamiento implementado}


El algoritmo de agrupaci�n implementado para la distribuci�n de las cadenas de caracteres es una variante del algoritmo CURE~\ref{Algoritmo de agrupamiento1}, que utiliza un algoritmo jer�rquicos para formar los grupos iniciales.

A continuaci�n se muestran las etapas que sigue el algoritmo implementado:

\begin{enumerate}
  \item El algoritmo empieza con la selecci�n de un algoritmo de agrupamiento jer�rquico aglomerativo o diviso, ambas opciones son v�lidas, en este caso se elige el aglomerativo. Luego, se obtiene una muestra peque�a de la colecci�n de datos, en lo posible la muestra debe ser lo suficientemente representativa de la colecci�n de datos. Posteriormente, se calcula la distancia, utilizando alguna medida de distancia para cadenas de caracteres entre todas las muestras (ver ~\ref{Medida de similitud}). Si las muestras se mantienen el algoritmo es determinista, es decir, siempre entregar� los mismos resultados, lo que implica una ventaja ya que puede replicarse en varias m�quinas.
  
 \item  Con las distancias de todas las muestras se  ejecuta el algoritmo de agrupamiento aglomerativo hasta obtener los grupos deseaos. Por �ltimo se elige en cada grupo los puntos representativos, a diferencia de CURE que selecciona algunos puntos, en este caso los puntos representativos son todos los puntos del grupo.

  \item Con los grupos construidos, se asigna cada cadena de caracteres de la colecci�n de datos al grupo que tenga el punto representativo  con la mejor medida de distancia. Uno de los objetivos de la memoria es el balance en t�rminos de la carga de almacenamiento de los grupos, para esto la cadena de caracteres, antes de ser asignado al grupo, se comprueba que~\ref{eq3123}

\begin{equation}
\begin{aligned}
  Grupo_{i} < \frac{C}{k},
   \end{aligned}\label{eq3123}
\end{equation}

donde $Grupo_{i}$ es el tama�o en disco del grupo m�s cercano a la cadena de caracteres seleccionada, $C$ es el tama�o en disco de la Colecci�n de cadenas de caracteres y $k$ es el n�mero de grupos.

Con esto, al momento de asignar una cadena de caracteres a un grupo, se busca generar un balance en cada grupo, si el $Grupo_{i}$ es mayor, entonces se comprueba el siguiente grupo m�s cercano a la cadena de caracteres seleccionada, hasta encontrar un grupo que sea menor. Si bien se genera un balance en el espacio de memoria la comprobaci�n se realiza antes de la compresi�n y puede ocurrir que al momento de la compresi�n no asegura un balance en todas las agrupaciones. Tambi�n puede pasar que un grupo contenga documentos que fueron asignados no por la medida de distancia sino por la falta de espacio en los dem�s grupos, lo que podri� provocar que el resultado de la compresi�n sea mayor al resto de los grupos.

\end{enumerate}


A continuaci�n se observa los pseucodigo del algoritmo implementado, el algoritmo~\ref{similitud} muestra la funci�n distancia que representa el resultado de la medida de distancia entre dos cadenas de caracteres y el algoritmo~\ref{alg1} muestra el algoritmo de agrupamiento propuesto en la memoria.

\begin{algorithm}
\begin{algorithmic}[1]
\REQUIRE $String1$
\REQUIRE $String2$
\STATE $s1 \leftarrow COMPRESS(String1)$
\STATE $s2 \leftarrow COMPRESS(String2)$
\STATE $s12 \leftarrow COMPRESS(String1+ String2)$
\RETURN $size(s12)-size(s2)/size(s1)$

\end{algorithmic}
\caption{Funcion DISTANCIA}\label{similitud}
\end{algorithm}


%------------------------------------%

\begin{algorithm}
\begin{algorithmic}[1]
\REQUIRE $Sampling=\{d_{1}, \dots, d_{n} \}$\COMMENT{Muestra obtenida de la colecci�n, donde $n$ es la cantidad de cadenas de caracteres.}
\REQUIRE $Collection=\{d_{1}, \dots, d_{k} \}$ \COMMENT{Colecci�n de datos, donde $k$ es numero de cadenas de caracteres.}
\REQUIRE $C$ \COMMENT{N�mero de grupos deseados.}
\STATE $S \leftarrow \langle\ \rangle$
\FOR{\textbf{each} s1 in $Sampling$}
\FOR{\textbf{each} s2 in $Sampling$} 
			\STATE $S \cup  (DISTANCIA(Sampling[s1],Sampling[s2]) , s1 ,s2)$
		\ENDFOR
\ENDFOR

\STATE Sort($S$)
\STATE $i = 0$
\WHILE {$Sampling$ > $C$}
	\STATE $ Sampling[S[i][1]] \cup Sampling[S[i][2]]$
	\STATE $i = i +1$
\ENDWHILE

\STATE $Grupo \leftarrow \langle\ \rangle$  \COMMENT{Grupo lista de tama�o $C$}
\FOR{\textbf{each} $d$ in $Collection$}
\STATE $i = 1$
\FOR{$s=0$ to $n$}
	\IF{$DISTANCIA(Sampling[s],d) < i$}
		\STATE $i = DISTANCIA(Sampling[s],d)$
	\ENDIF
\ENDFOR
\STATE $Grupo[i] \cup d$
\ENDFOR

\end{algorithmic}
\caption{Algoritmo de agrupamiento propuesto}\label{alg1} 
\end{algorithm}


\section{Mejoras del Algoritmo de agrupamiento implementado}\label{mejoras123}

\subsection{Paralelizaci�n}

Al aumentar la muestra el tiempo de ejecuci�n crece $\frac{n^3}{2}$, donde $n$ es la cantidad de documentos. Como el objetivo de la memoria est� orientado al manejo de datos de gran volumen, es importante que el algoritmo propuesto pueda ejecutarse en tiempos razonables. La mayor parte del tiempo de ejecuci�n del algoritmo propuesto se utiliza para calcular las distancias entre las cadenas de caracteres.

Una de las ventajas del algoritmo de agrupamiento implementado permite la paralelizaci�n, es decir, ejecutar varios procesos paralelos en los que se consume el mayor tiempo de ejecuci�n. Esto ocurre cuando se calculan las distancias entre las muestras para generar los grupos iniciales y al calcular las distancias entre los documentos que ser�n asignados con las muestras. En el algoritmo \ref{alg4} muestra las modificaciones necesarias para poder implementar el algoritmo con paralelizaci�n.

\begin{algorithm}
\begin{algorithmic}[1]
\REQUIRE $Sampling=\{d_{1}, \dots, d_{n} \}$\COMMENT{Muestra obtenida de la colecci�n, donde $n$ es la cantidad de cadenas de caracteres.}
\REQUIRE $Collection=\{d_{1}, \dots, d_{k} \}$ \COMMENT{Colecci�n de datos, donde $k$ es numero de cadenas de caracteres.}
\REQUIRE $C$ \COMMENT{N�mero de grupos deseados.}
\STATE $S \leftarrow \langle\ \rangle$

	\STATE $S \cup PARALELIZAR( (DISTANCIA(Sampling[s1],Sampling[s2]) , s1 ,s2), num_procesos )$

\STATE Sort($S$)
\STATE $i = 0$
\WHILE {$Sampling$ > $C$}
	\STATE $ Sampling[S[i][1]] \cup Sampling[S[i][2]]$
	\STATE $i = i +1$
\ENDWHILE

\STATE $Grupo \leftarrow \langle\ \rangle$  \COMMENT{Grupo lista de tama�o $C$}
\STATE $PARALELIZAR \{$
\STATE $i = 0$
\FOR{$s=0$ to $n$}
	\IF{$DISTANCIA(Sampling[s],d) > i$}
		\STATE $i = DISTANCIA(Sampling[s],d)$
	\ENDIF
\ENDFOR
\STATE $Grupo[i] \cup d$
\STATE $\}$

\end{algorithmic}
\caption{Algoritmo de agrupamiento propuesto Paralelizaci�n}\label{alg4} 
\end{algorithm}

La mejora en los tiempos de ejecuci�n est� directamente relacionado con la cantidad de procesos que se quiera ejecutar, estos deben ser menor o igual a la cantidad de procesadores que cuenta la m�quina en la que se ejecutar� el algoritmo. El tiempo de ejecuci�n se divide por cada proceso adicional.

Otra manera para mejorar el tiempo de ejecuci�n es seleccionando algunas cadenas de caracteres como puntos representativos, pero se debe buscar el modo que los puntos representativos realmente representen todos las cadenas de caracteres del grupo. En el caso del espacio euclidiano es f�cil de lograr, pero para cadenas de caracteres es muy dif�cil, ya que no asegura que representen todas las cadenas. Podr�a pasar el caso de que todas las cadenas de caracteres est�n muy alejadas entre ellas, lo cual significa que todos deber�an ser puntos representativos, por eso el algoritmo propuesto asegura todas las cadenas de caracteres del grupo como puntos representativos sacrificando tiempo de ejecuci�n.

\subsection{Estrategia Greedy}

Una posible mejora que se puede emplear al algoritmo de agrupamiento propuesto es al problema que se genera cuando se asignan los documentos a un grupo con la restricci�n de un m�ximo de documentos asignados a cada grupo, utilizado para balancear la carga de estos. El algoritmo selecciona un documento y se asigna a un grupo, sin tomar en cuenta los documentos que ser�n asignados posteriormente. En consecuencia, el documento asignado a un grupo podr�a no ser la mejor opci�n frente a un posible candidato con una mejor distancia para ese mismo grupo.

Para solucionar este problema se puede utilizar la estrategia greedy, lo que significa que siempre toma la mejor opci�n local para lograr la mejor soluci�n, aunque no implica que siempre llegue a la mejor soluci�n. La mejora se implement� ocupando una cola de prioridad, basado en la distancias de cada documento a todos los grupos, para luego asignar los documentos en el orden que la cola de prioridad entrega.

El algoritmo \ref{alg5} muestra las modificaciones necesarias para poder implementar esta mejora.

%------------------------------------%

\begin{algorithm}
\begin{algorithmic}[1]
\REQUIRE $Sampling=\{d_{1}, \dots, d_{n} \}$\COMMENT{Muestra obtenida de la colecci�n, donde $n$ es la cantidad de cadenas de caracteres.}
\REQUIRE $Collection=\{d_{1}, \dots, d_{k} \}$ \COMMENT{Colecci�n de datos, donde $k$ es numero de cadenas de caracteres.}
\REQUIRE $C$ \COMMENT{N�mero de grupos deseados.}
\STATE $S \leftarrow \langle\ \rangle$
\FOR{\textbf{each} s1 in $Sampling$}
\FOR{\textbf{each} s2 in $Sampling$} 
			\STATE $S \cup  (DISTANCIA(Sampling[s1],Sampling[s2]) , s1 ,s2)$
		\ENDFOR
\ENDFOR

\STATE Sort($S$)
\STATE $i = 0$
\WHILE {$Sampling$ > $C$}
	\STATE $ Sampling[S[i][1]] \cup Sampling[S[i][2]]$
	\STATE $i = i +1$
\ENDWHILE



\STATE $Grupo \leftarrow \langle\ \rangle$  \COMMENT{Grupo lista de tama�o $C$}
\STATE $Cola_prioridad \leftarrow \langle\ \rangle$ 

\FOR{\textbf{each} $d$ in $Collection$}
\FOR{$s=0$ to $n$}
	\STATE $Cola_prioridad.add(DISTANCIA(Sampling[s],d),d,s)$	
\ENDFOR
\ENDFOR

\WHILE {$Cola_prioridad$}
	\STATE $i = Cola_prioridad.pop()$
	\STATE $Grupo[i[2]] \cup i$
\ENDWHILE

\end{algorithmic}
\caption{Algoritmo de agrupamiento propuesto Estrategia Greedy }\label{alg5} 
\end{algorithm}

\chapter[Resultados preliminares ]{Resultados }\label{ch:capitulo2}


En las pruebas se tom� una muestra no superior a 30 documentos ya que al aumentar la muestra el tiempo de ejecuci�n crece $\frac{n^3}{2}$, donde $n$ es la cantidad de documentos. La muestra se obtuvo aleatoriamente de la colecci�n de documentos, como la muestra es insignificante en comparaci�n al tama�o de la muestra es muy probable que ning�n documentos perteneciera a un documento de la misma versi�n. Este problema origina que la mayor�a de los documentos pertenezca a un solo grupo y el resto solamente es representado por un documento. Tambi�n cabe mencionar que la elecci�n de la cantidad de agrupaciones es arbitraria, pero la cantidad de agrupaciones es una variable importante al momento de obtener buenos resultados en las agrupaciones, en este caso las pruebas se realizaron con un n�mero fijo de agrupaciones para observar el comportamientos de otras variables que afectan a las agrupaciones.

En la tabla \ref{Resultado algoritmo de agrupamiento aleatorio con 10 Grupos.}  muestra los resultados  de cada m�todo con una cantidad de 10 agrupaciones. El \textit{M�todo 1}  utiliza el algoritmo de agrupaci�n aleatoria, se tiene que en cada grupo se mantiene una carga de almacenamiento balanceada que es uno de los objetivos deseados en la memoria. En los m�todos siguientes se utiliza el algoritmo de agrupaci�n propuesto pero modificando algunas variables para observar su comportamiento.

Para el caso del \textit{M�todo 2}  se observa una mejora de la compresi�n equivalente al  45\% del tama�o total del resultado en el algoritmo de agrupamiento aleatorio, aqu� la muestra es de 30 documentos. En t�rminos de balance en la carga de almacenamiento que se representa en el $Error$ de la tabla~\ref{distribucion}, este m�todo es ineficiente ya que la mayor parte de la carga se concentra solamente en un grupo. Esto se debe a que en el momento de crear los grupos con las muestras, la mayor parte de las muestra quedan solamente en un grupo dejando a las dem�s con pocas muestras de representaci�n.

En el \textit{M�todo 3} puede apreciar que existe un balance en la cantidad de muestras en cada agrupaci�n. Para esto, cada grupo no tendr� una muestra de documentos superior a 3 en un universo de 30 documentos. Con esto se busca balancear la cantidad de documentos en cada agrupaci�n.
El resultado de la compresi�n utilizando el \textit{M�todo 3} es equivalente al 55\% del tama�o total del resultado con el m�todo del algoritmo de agrupamiento aleatorio, que sigue siendo una mejor alternativa, pero comparando con los resultado del \textit{M�todo 2} se paga un costo al balancear las muestras en los grupos, de un 24\% m�s del tama�o total del resultado en el \textit{M�todo 2}.

En el \textit{M�todo 4}, se hace la misma prueba que en el m�todo anterior  pero se agrega la condici�n de que el tama�o de los grupos no supere un l�mite. El l�mite en este caso es el tama�o de la colecci�n de datos divido por la cantidad de agrupaciones, con esta medida se asegura que en todos los grupos tengan aproximadamente la misma cantidad de cadenas de caracteres.  El resultado del \textit{M�todo 4} es el equivalente al 70\% del tama�o total del resultado en el algoritmo de agrupamiento aleatorio que sigue siendo una mejora, pero nuevamente pagando un costo, con respecto al \textit{M�todo 2} aumenta 55\% m�s de tama�o, incluso mayor que en el \textit{M�todo 3}, pero con mejores resultados en el balance de la carga de almacenamiento. 


%------------------------------------------------------%


\begin{table}[H]
\begin{center}
\resizebox{15cm}{!} {

\begin{tabular}{|p{3cm}|p{3cm}||p{3cm}||p{3cm}||p{3cm}|}

\hline
Grupos & M�todo 1(KiB)  & M�todo 2(KiB)  & M�todo 3(KiB)  & M�todo 4(KiB) \\
\hline
Total  & 65.510 & 29.371 (45\%) & 36.036 (55\%) & 45.639 (70\%) \\
\hline
\end{tabular}
}
\end{center}
\caption{Resultado algoritmo de agrupamiento aleatorio con 10 Grupos.}

\label{Resultado algoritmo de agrupamiento aleatorio con 10 Grupos.}

\end{table}	

En la tabla ~\ref{distribucion} se observa el resultado de la distribucion de los datos de cada grupo utilizando los metodos mencionados. $Error$ representa el promedio del error absoluto en cada grupo, $Tama�o$ representa el promedio de la carga de almacenamiento que ocupan los grupos y $Similitud$ es la similitud entre los documentos de las muestras de un mismo grupo.


\begin{table}[H]
\begin{center}
\resizebox{15cm}{!} {

\begin{tabular}{|p{3cm}|p{3cm}||p{3cm}||p{3cm}|}

\hline
Grupos & Error  & Tama�o(KiB)  & Similitud \\
\hline
M�todo 1 &  0.2  & 6553.6 &  - \\
\hline
M�todo 2 & 1910.4  & 2937.2 & 0.196718 \\
\hline
M�todo 3 & 509 & 3603.7 & 0.994174 \\
\hline
M�todo 4 & 1.4 & 4563.9 & 0.994174  \\
\hline
\end{tabular}
}
\end{center}
\caption{Distribuci�n grupos.}

\label{distribucion}

\end{table}	




\chapter[Conclusi�n ]{Conclusi�n}\label{ch:capitulo4}
\fpar

Se puede concluir que agrupando las secuencias de string de manera inteligente se obtiene mejores resultados que agrupandolos aleatoriamente, pero intentar balancear la carga de espacio en cada agrupacion se paga un costo al comprimir. Parte importante para obtener una buena agrupacion es la medida de similitud , para agrupar grandes cantidades de string es necesario que la medida de similitud entre dos string sea rapido ya que cada secuencia de string debe compararse con cada secuencia del sampling. La ventaja de este algoritmo son que los resultados son determinista, es decir, cuantas veces se ejecuta el algoritmo para una misma coleccion siempre entrega el mismo resulatdo, entonces al momento de obtener las agrupaciones con el sampling es posible asignar string en varios procesos rebajando el tiempo de ejecucion.

El plan de trabajo para la segunda etapa de la tesis consistira en intentar mejorar los resultados de la compresion de la coleccion de datos obtenidos en esta primera parte , por ejemplo, cambiando las distintas variables que influyen en agrupacion de strings como el numero de agrupaciones , para obtener mejores resultados en la compresion de las agrupaciones.

% ... mas archivos de capitulos
%
% ---------------------------------------------------------------
% Bibliograf�a
% ---------------------------------------------------------------
% tubiblio.bib es el archivo con la base de datos bibliografica
%
\bibliographystyle{unsrt}
\bibliography{udpthesis}
%
% ---------------------------------------------------------------
% Simbolog�a y glosario
% ---------------------------------------------------------------
% simbolos.tex es el archivo de simbolos (y glosario)
%
%\begin{symbology}
%\input{simbolos}  % archivo propio de simbolos
%\end{symbology}
%
% ---------------------------------------------------------------
% Anexos
% ---------------------------------------------------------------
\appendix
%
% aca se incluyen los archivos con el texto de los anexos
% (Ej.: anx-uno.tex es el archivo de un anexo)
%



%==================================================ANEXOS========================================%
\chapter{CODIGO ALGORITMO DE AGRUPAMIENTO BASE}
\label{ch:Codigo Metodo de agrupación propuesto}



%\begin{minted}{python}
\begin{python}







#!/usr/bin/python

""" argv[1]= carpeta de documetnos para generar clusters
	argv[2]= Numero de clusters deseado
	argv[3]= carpeta guardar clusters
	argv[4]= carpeta de documentos
	argv[5]= max cluster size
	argv[6]= calidad distancia documento 0-9
"""
import os, sys ,getopt
import editdist
import shutil
import distance
import zlib


def get_size(start_path = '.'):
    total_size = 0
    for dirpath, dirnames, filenames in os.walk(start_path):
        for f in filenames:
            fp = os.path.join(dirpath, f)
            total_size += os.path.getsize(fp)
    return total_size

def comprimir( ):
	for i in os.listdir(str(sys.argv[3])):
		os.system('7z a '+i+'.7z '+str(sys.argv[3])+i+'/')
	
def distancia_zip(s1, s2, nivel=6):
	compressed1 = zlib.compress(s1,nivel)
	compressed2 = zlib.compress(s2,nivel)
	compressed12 = zlib.compress(s1+s2,nivel)
	#c1=float(len(compressed1))
	#c2=float(len(compressed2))
	#c12=float(len(compressed12))
	#return float(c12/(c2+c1))
	if len(compressed1) > len(compressed2):
		n = (len(compressed12) - len(compressed2)) / float(len(compressed1))
	else:
		n = (len(compressed12) - len (compressed1)) / float(len(compressed2))	
	return n

		
    
if __name__ == "__main__":
	
	#numero de clusters
	clusters = int(sys.argv[2])
	#doculmentos para generar los clusters
	data=[]  
	#guardar bloque de datos 
	for document in os.listdir(str(sys.argv[1])):
		f = open(str(sys.argv[1])+str(document))
		data.append([f.read()])
		f.close()


	#lista documentos ordenados por sus distancias 
	test=[]
	print "sampling cargado, buscando las distancias minimas..."
	for k in range(len(data)):
			for j in range(k+1,len(data)):
				comp=distancia_zip(data[k][0],data[j][0],int(sys.argv[6]))
				print "t1:"+str(k)+" contra t2:"+str(j)+"="+str(comp)
				test.append([comp,data[k][0],data[j][0]])          
	test.sort()
	
	print len(test)	
		
		
	print "creando clusters"
	i=0
	while len(data)>1 and len(data) > clusters and i < len(test):
		cluster1=[]
		cluster2=[]
		t1=False
		t2=False
		for j in data:
			if t1 == True and t2 == True :
				break
			for k in range(len(j)):
				if test[i][1] == j[k]:
					t1=True
					cluster1=data.index(j)
				if test[i][2] == j[k]:
					t2= True
					cluster2=data.index(j)
		i=i+1
		if cluster2 ==cluster1 or (len(data[cluster1])+len(data[cluster2])) > int(sys.argv[5]) : 
			continue
		test3=data.pop(cluster2)		
		if cluster2<cluster1:
			cluster1-=1
				
		test4=data.pop(cluster1)
		data.append(test3+test4)
		
		densidad=[]		
		print "densididad sampling clster:"+str(test[i][0])
		for t in data:
			densidad.append(len(t))
		print densidad
		
	densidad=[]		
	print "IMPRIMIR CLUSTER"
	for t in data:
		densidad.append(len(t))
	print densidad
	
	print "Creando carpetas de clusters..."
	
	clust_size=[]
	for t in range(len(data)):
		os.mkdir( sys.argv[3]+str(t));
		clust_size.append(0)

	div=get_size(str(sys.argv[4]))/len(data)
		
	i=0
	
	print "Asignando documentos a los clusters"
	for document in os.listdir(str(sys.argv[4])):		
		test2=[]
		f = open(str(sys.argv[4])+str(document))		
		texto=f.read()
		doc_z = len(texto)
		for j in data:
			for k in range(len(j)):
				#comp=editdist.distance(j[k],texto)
				#comp=distance.hamming(j[k],texto)	
				comp=distancia_zip(j[k],texto,int(sys.argv[6]))	
				test2.append([comp,data.index(j)])	          
		test2.sort()
		
		for s in test2:
			if clust_size[s[1]] < div:
				shutil.copyfile(str(sys.argv[4])+str(document), sys.argv[3]+str(s[1])+'/'+str(document))
				clust_size[s[1]]+=doc_z
				break
		f.close()
	
	for s in clust_size:
		print s
	
	comprimir()	
	
	

\end{python}
%\end{minted}


\chapter{CODIGO ALGORITMO DE AGRUPAMIENTO ALEATORIO}
\label{ch:Codigo Metodo de agrupación random}

\begin{python}
#!/usr/bin/python
import glob
import os,sys
import random
import shutil

""" argv[1]= carpeta de documentos
	argv[2]= n clusters
	argv[3]= carpeta resultados
"""

print sys.argv[1]
lista=os.listdir(str(sys.argv[1]))
random.seed()
print len(lista)


print "Creando carpetas de clusters..."
	
for t in range(int(sys.argv[2])):
	os.mkdir( sys.argv[3]+str(t));

div=len(lista)/int(sys.argv[2])
print div



for i in range(int(sys.argv[2])) :
	for j in range(div) :	
		rand = random.randint(0,len(lista)-1)	
		texto= lista.pop(rand)	
		shutil.copyfile( sys.argv[1]+texto , sys.argv[3]+str(i)+"/"+texto)
		
		



\end{python}







% ... mas archivos de anexos
%
% ---------------------------------------------------------------
% Fin del documento
% NO ESCRIBIR DESPU�S DE ESTA LINEA
\backmatter
\end{document}
% ---------------------------------------------------------------
